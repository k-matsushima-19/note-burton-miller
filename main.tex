%%%%%%%%%%%%%%%%%%%%%%%%%%%%%%%%%%%%%%%%%%%%%%%%%%%%%%%%%%%%%%%
%
% Welcome to Overleaf --- just edit your LaTeX on the left,
% and we'll compile it for you on the right. If you open the
% 'Share' menu, you can invite other users to edit at the same
% time. See www.overleaf.com/learn for more info. Enjoy!
%
%%%%%%%%%%%%%%%%%%%%%%%%%%%%%%%%%%%%%%%%%%%%%%%%%%%%%%%%%%%%%%%
\documentclass{article}
\usepackage{amsmath}
\usepackage{amsfonts}
\usepackage{amsthm}
\usepackage{hyperref}


\theoremstyle{plain}
\newtheorem{thm}{Theorem}[section]
\newtheorem*{thm*}{Theorem}

\newtheorem{lemma}[thm]{Lemma}
\newtheorem*{lemma*}{Lemma}

\newtheorem{prop}[thm]{Proposition}
\newtheorem*{prop*}{Proposition}

\newtheorem{cor}[thm]{Corollary}
\newtheorem*{cor*}{Corollary}

\theoremstyle{definition}
\newtheorem{dfn}[thm]{Definition}

% cleveref
\usepackage{cleveref}
\crefname{equation}{}{}
\crefname{thm}{Theorem}{Theorem}

\title{On the solvability of Burton--Miller-type boundary integral equations}
\author{Kei MATSUSHIMA}
\begin{document}
\maketitle

\section{Exterior problems}
Throughout this note, we assume that $\Omega$ is a bounded subset of $\mathbb R^d$ ($d=2$ or $3$) which is the open complement of an unbounded domain of class $C^2$. Namely, the open and bounded subset $\Omega$ is not necessarily connected.

Given a constant $k\in\mathbb C$, we shall consider
\begin{itemize}
    \item Exterior Dirichlet problem: find $u\in C^2(\mathbb R^d\setminus\overline\Omega) \cap C^1(\mathbb R^d\setminus\Omega)$ such that
    \begin{align}\label{eq:dirichlet}
        \begin{cases}
            \varDelta u + k^2 u = 0 \quad &\text{ in } \Omega,\\
            u = f \quad &\text{ on } \partial \Omega,
            \\
            \displaystyle r^{(d-1)/2}\left( \frac{\partial u}{\partial r} - \mathrm iku\right) \to 0  \quad &\text{ uniformly as } |x|=:r\to\infty
        \end{cases}
    \end{align}
    \item Exterior Neumann problem: find $u\in C^2(\mathbb R^d\setminus\overline\Omega) \cap C^1(\mathbb R^d\setminus\Omega)$ such that
    \begin{align}\label{eq:neumann}
        \begin{cases}
            \varDelta u + k^2 u = 0 \quad &\text{ in } \Omega,\\
            \displaystyle \left. \frac{\partial u}{\partial \nu}\right|_+ = g \quad &\text{ on } \partial \Omega,
            \\
            \displaystyle r^{(d-1)/2}\left( \frac{\partial u}{\partial r} - \mathrm iku\right) \to 0  \quad &\text{ uniformly as } |x|=:r\to\infty
        \end{cases}
        ,
    \end{align}
\end{itemize}
where $f$ and $g$ are given functions on $\partial\Omega$. The normal derivative should be understood in the sense of uniform convergence with parallel surfaces. For a function $\varphi\in C^1(\Omega)$, we define
\begin{align*}
    \left. \frac{\partial \varphi}{\partial \nu}\right|_-(x) := \lim_{h\to +0} \nu(x)\cdot \nabla \varphi(x - h\nu(x)) \quad\text{ on }\partial\Omega
\end{align*}
when it converges uniformly on $\partial\Omega$, where $\nu$ is the unit normal vector on $\partial\Omega$ outward to $\Omega$. Similarly, for a function $\varphi\in C^1(\mathbb R^d\setminus\overline\Omega)$, we define
\begin{align*}
    \left. \frac{\partial \varphi}{\partial \nu}\right|_+(x) := \lim_{h\to +0} \nu(x)\cdot \nabla \varphi(x + h\nu(x)) \quad\text{ on }\partial\Omega
\end{align*}
when it converges uniformly on $\partial\Omega$.

The well-posedness of the exterior Dirichlet and Neumann problems can be established via layer-potential techniques. 
\begin{thm}[Colton and Kress \cite{colton1998inverse}, Theorem 3.11]\label{thm:wellposed-dirichlet}
    Let $k>0$ and $f\in C(\partial\Omega)$. Then the exterior Dirichlet problem \cref{eq:dirichlet} is uniquely solvable.
\end{thm}

\begin{thm}[Colton and Kress \cite{colton1998inverse}, Theorem 3.12]\label{thm:wellposed-neumann}
    Let $k>0$ and $g\in C(\partial\Omega)$. Then the exterior Neumann problem \cref{eq:neumann} is uniquely solvable.
\end{thm}

As a consequence, we immediately obtain the following results.
\begin{cor}
    Suppose that a function $\varphi\in C^2(\mathbb R^d\setminus\overline\Omega)$ solves the Helmholtz equation
    \begin{align*}
        \varDelta \varphi + k^2 \varphi  = 0 \quad\text{ in }\mathbb R^d\setminus\overline\Omega
    \end{align*}
    with radiation condition
    \begin{align*}
        r^{(d-1)/2}\left( \frac{\partial \varphi}{\partial r} - \mathrm ik\varphi\right) \to 0  \quad &\text{ uniformly as } |x|=:r\to\infty
    \end{align*}
    for some $k>0$ and 
    \begin{align}\label{eq:tmp1}
        \lim_{h\to+0} \varphi (x+h\nu(x)) = 0 
    \end{align}
    uniformly on $\partial\Omega$. Then $\varphi (x)=0$ for all $x\in \mathbb R^d\setminus\overline\Omega$.
\end{cor}
\begin{proof}
    Define a function $u$ by
    \begin{align*}
        u(x) :=
        \begin{cases}
            \varphi(x) & x\in \mathbb R^d\setminus\overline\Omega
            \\
            0 & x\in \partial\Omega
        \end{cases}
        .
    \end{align*}
    It is easy to see that $u$ belongs to $C^2(\mathbb R^d\setminus\overline\Omega) \cap C^1(\mathbb R^d\setminus\Omega)$ from the uniform convergence \cref{eq:tmp1}. Since $u\in C^2(\mathbb R^d\setminus\overline\Omega) \cap C^1(\mathbb R^d\setminus\Omega)$ solves the exterior Dirichlet problem \cref{eq:dirichlet}, the unique solvability (\cref{thm:wellposed-dirichlet}) ensures that $u=0$ in $\mathbb R^d\setminus\overline\Omega$.
\end{proof}

\begin{cor}
    Suppose that a function $\varphi\in C^2(\mathbb R^d\setminus\overline\Omega)$ solves the Helmholtz equation
    \begin{align*}
        \varDelta \varphi + k^2 \varphi  = 0 \quad\text{ in }\mathbb R^d\setminus\overline\Omega
    \end{align*}
    with radiation condition
    \begin{align*}
        r^{(d-1)/2}\left( \frac{\partial \varphi}{\partial r} - \mathrm ik\varphi\right) \to 0  \quad &\text{ uniformly as } |x|=:r\to\infty
    \end{align*}
    for some $k>0$ and 
    \begin{align}\label{eq:tmp1}
        \left.\frac{\partial \varphi}{\partial\nu}\right|_+ 
 := \lim_{h\to+0} \nu(x)\cdot \nabla\varphi (x+h\nu(x)) = 0 
    \end{align}
    uniformly on $\partial\Omega$. Then $\varphi (x)=0$ for all $x\in \mathbb R^d\setminus\overline\Omega$.
\end{cor}
\begin{proof}
    It is easy to see that there exists a unique extension $u\in C^2(\mathbb R^d\setminus\overline\Omega) \cap C^1(\mathbb R^d\setminus\Omega)$ of $\varphi\in C^2(\mathbb R^d\setminus\overline\Omega)$ such that $\varphi = u$ in $\mathbb R^d\setminus\overline\Omega$ with the uniform convergence 
    \begin{align*}
        \left.\frac{\partial u}{\partial\nu}\right|_+ = \lim_{h\to+0} \nu(x)\cdot \nabla u (x+h\nu(x)) = 0 .
    \end{align*}
    The statement immediately follows from \cref{thm:wellposed-neumann}, i.e., the unique solvability of the exterior Neumann problem \cref{eq:neumann}.
\end{proof}

\section{Layer potentials and boundary integral operators}
Let $G$ be the fundamental solution of the Helmholtz equation, given by
\begin{align*}
    G(x,y) = \frac{\mathrm i}{4}H^{(1)}_0(k|x-y|), \quad x,y\in\mathbb R^2,x\neq y
\end{align*}
for $d=2$ or
\begin{align*}
    G(x,y) = \frac{1}{4\pi} \frac{\mathrm e^{\mathrm ik|x-y|}}{|x-y|}, \quad x,y\in\mathbb R^3,x\neq y
\end{align*}
for $d=3$, where $H^{(1)}_n$ is the Hankel function of the first kind and order $n$. Given $\varphi\in C(\partial\Omega)$, the function
\begin{align*}
    \mathbb R^d\setminus \partial\Omega \ni x\mapsto \int_{\partial\Omega} G(x,y)\varphi(y)\mathrm ds(y)
\end{align*}
is called a \textit{single-layer potential} with density $\varphi\in C(\partial\Omega)$. Analogously, for $\varphi\in C(\partial\Omega)$, 
\begin{align*}
    \mathbb R^d\setminus \partial\Omega \ni x\mapsto \int_{\partial\Omega} \frac{\partial G}{\partial \nu(y)}(x,y)\varphi(y)\mathrm ds(y)
\end{align*}
is called a \textit{double-layer potential} with density $\varphi\in C(\partial\Omega)$.

We are interested in the limiting case where the point $x$ lies on the boundary $\partial\Omega$. In view of this, we formally write
\begin{align*}
    (S\varphi)(x) :=& \int_{\partial\Omega} G(x,y)\varphi(y) \mathrm ds(y) \quad x\in\partial\Omega,
\\
    (D\varphi)(x) :=& \int_{\partial\Omega} \frac{\partial G}{\partial \nu(y)}(x,y)\varphi(y) \mathrm ds(y) \quad x\in\partial\Omega,
\\
    (D^\prime \varphi)(x) :=& \int_{\partial\Omega} \frac{\partial G}{\partial \nu(x)}(x,y)\varphi(y) \mathrm ds(y) \quad x\in\partial\Omega,
\\
    (N\varphi)(x) :=& \frac{\partial}{\partial\nu(x)} \int_{\partial\Omega} \frac{\partial G}{\partial \nu(y)}(x,y)\varphi(y) \mathrm ds(y) \quad x\in\partial\Omega.
\end{align*}
In order that the operators $S$, $D$, $D^\prime$, and $N$ are well-defined, the function $\varphi$ requires sufficient regularity on $\partial\Omega$. We have the following \textit{mapping properties} of boundary integral operators:
\begin{thm}[Colton and Kress \cite{colton1998inverse}, Theorem 3.4]
    Let $k>0$, $\alpha\in(0,1]$ and let $\partial\Omega$ be of class $C^2$. Then the following operators are well-defined and linear with some additional properties:
    \begin{itemize}
        \item $S:C(\partial\Omega)\to C(\partial\Omega)$, compact
        \item $S:C(\partial\Omega)\to C^{0,\alpha}(\partial\Omega)$, bounded
        \item $S:C^{0,\alpha}(\partial\Omega)\to C^{1,\alpha}(\partial\Omega)$, bounded
        \item $S:C^{1,\alpha}(\partial\Omega)\to C^{1,\alpha}(\partial\Omega)$, compact
        \item $D:C(\partial\Omega)\to C(\partial\Omega)$, compact
        \item $D:C(\partial\Omega)\to C^{0,\alpha}(\partial\Omega)$, bounded
        \item $D:C^{0,\alpha}(\partial\Omega)\to C^{1,\alpha}(\partial\Omega)$, bounded
        \item $D:C^{1,\alpha}(\partial\Omega)\to C^{1,\alpha}(\partial\Omega)$, compact
        \item $D^\prime:C(\partial\Omega)\to C(\partial\Omega)$, compact
        \item $D^\prime:C(\partial\Omega)\to C^{0,\alpha}(\partial\Omega)$, bounded
        \item $D^\prime:C^{1,\alpha}(\partial\Omega)\to C^{1,\alpha}(\partial\Omega)$, compact
        \item $N:C^{1,\alpha}(\partial\Omega)\to C^{0,\alpha}(\partial\Omega)$, bounded
        \item $SN:C^{1,\alpha}(\partial\Omega)\to C^{1,\alpha}(\partial\Omega)$, $SN=D^2-I$, Fredholm of index zero
        \item $NS:C^{0,\alpha}(\partial\Omega)\to C^{0,\alpha}(\partial\Omega)$, $NS=(D^\prime)^2 - I$, Fredholm of index zero
    \end{itemize}
\end{thm}

The layer potentials are associated with the boundary integral operators via the following \textit{jump relations}:
\begin{thm}[Colton and Kress \cite{colton1998inverse}, Theorem 3.1]
    Let $k>0$ and $\varphi\in C(\partial\Omega)$. Then the single-layer potential
    \begin{align*}
        w(x) := \int_{\partial\Omega} G(x,y)\varphi(y)\mathrm ds(y) \quad x\in\mathbb R^d\setminus\partial\Omega
    \end{align*}
    is twice-continuously differentiable and solves the Helmholtz equation
    \begin{align*}
        \varDelta w(x) + k^2 w(x) = 0 \quad \text{ for all }x\in \mathbb R^d\setminus\partial\Omega
    \end{align*}
    with radiation condition
    \begin{align*}
        r^{(d-1)/2}\left( \frac{\partial w}{\partial r} - \mathrm ikw\right) \to 0  \quad &\text{ uniformly as } |x|=:r\to\infty.
    \end{align*}
    Moreover, the limit
    \begin{align*}
        \lim_{h\to +0} w(x \pm h\nu(x)) = (S\varphi)(x)
    \end{align*}
    converges uniformly to the continuous function on $\partial\Omega$. Analogously, the limit
    \begin{align*}
        \left.\frac{\partial w}{\partial\nu}\right|_\pm(x) :=  \lim_{h\to +0} \nu(x)\cdot \nabla w(x \pm h\nu(x)) = \mp \frac{1}{2}\varphi(x) + (D^\prime\varphi)(x)
    \end{align*}
    converges uniformly to the continuous function on $\partial\Omega$.
\end{thm}

\begin{thm}[Colton and Kress \cite{colton1998inverse}, Theorem 3.1]
    Let $k>0$ and $\varphi\in C(\partial\Omega)$. Then the double-layer potential
    \begin{align*}
        v(x) := \int_{\partial\Omega} \frac{\partial G}{\partial \nu(y)}(x,y)\varphi(y)\mathrm ds(y) \quad x\in\mathbb R^d\setminus\partial\Omega
    \end{align*}
    is twice-continuously differentiable and solves the Helmholtz equation
    \begin{align*}
        \varDelta v(x) + k^2 v(x) = 0 \quad \text{ for all }x\in \mathbb R^d\setminus\partial\Omega
    \end{align*}
    with radiation condition
    \begin{align*}
        r^{(d-1)/2}\left( \frac{\partial v}{\partial r} - \mathrm ikv\right) \to 0  \quad &\text{ uniformly as } |x|=:r\to\infty.
    \end{align*}
    Moreover, the limit
    \begin{align*}
        \lim_{h\to +0} v(x \pm h\nu(x)) = \pm \frac{1}{2}\varphi(x) + (D\varphi)(x)
    \end{align*}
    converges uniformly to the continuous function on $\partial\Omega$, and
    \begin{align*}
        \lim_{h\to +0} \left[
            \nu(x)\cdot \nabla v(x + h\nu(x)) - \nu(x)\cdot \nabla v(x - h\nu(x))
        \right] = 0
    \end{align*}
    uniformly on $\partial\Omega$.
\end{thm}

\bibliographystyle{unsrt}
\bibliography{ref}

\end{document}


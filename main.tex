%%%%%%%%%%%%%%%%%%%%%%%%%%%%%%%%%%%%%%%%%%%%%%%%%%%%%%%%%%%%%%%
%
% Welcome to Overleaf --- just edit your LaTeX on the left,
% and we'll compile it for you on the right. If you open the
% 'Share' menu, you can invite other users to edit at the same
% time. See www.overleaf.com/learn for more info. Enjoy!
%
%%%%%%%%%%%%%%%%%%%%%%%%%%%%%%%%%%%%%%%%%%%%%%%%%%%%%%%%%%%%%%%
\documentclass{article}
\usepackage{amsmath}
\usepackage{amsfonts}
\usepackage{amsthm}
\usepackage{hyperref}


\theoremstyle{plain}
\newtheorem{thm}{Theorem}
\newtheorem*{thm*}{Theorem}
\newtheorem{lemma}{Lemma}
\newtheorem*{lemma*}{Lemma}
\newtheorem{prop}{Proposition}
\newtheorem*{prop*}{Proposition}

\theoremstyle{definition}
\newtheorem{dfn}{Definition}

% cleveref
\usepackage{cleveref}

\title{On the solvability of Burton--Miller-type boundary integral equations}
\author{Kei MATSUSHIMA}
\begin{document}
\maketitle

\section{Scattering problems}
Throughout this note, we assume that $\Omega$ is a bounded subset of $\mathbb R^d$ ($d=2$ or $3$) which is the open complement of an unbounded domain of class $C^2$. Namely, the open and bounded subset $\Omega$ is not necessarily connected.

Given a constant $k>0$, we shall consider
\begin{align}\label{eq:gE}
\begin{cases}
\varDelta u + k^2 u = 0 \quad &\text{ in } \Omega,\\
u = 0
\quad &\text{ on } \partial \Omega,
\end{cases}
\end{align}

\end{document}